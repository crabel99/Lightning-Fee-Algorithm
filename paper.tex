\documentclass[review,12pt]{elsarticle}
\usepackage[left=3.2cm, right=3.2cm, top=3.2cm, bottom=3.2cm]{geometry}

\usepackage{lineno, hyperref}
\modulolinenumbers[5]

% \journal{Theoretical Economics}
\usepackage{graphicx}
% \usepackage[abbr]{harvard}
\usepackage{amsmath}
% \usepackage{amsthm}
\usepackage{amsfonts}
\usepackage{amssymb}
\usepackage{braket}
\usepackage{threeparttable}
\usepackage[table]{xcolor}

% Solution to fix compiler warning https://tex.stackexchange.com/a/515199
\usepackage{regexpatch}
\makeatletter
\regexpatchcmd\ps@pprintTitle
  {\cE\}\ \cE\}}{\cE\}\cE\}}
  {}{\FailedToPatch}
\makeatother

% \RequirePackage[colorlinks,citecolor=blue,linkcolor=blue,urlcolor=blue,pagebackref]{hyperref}

%%%%%%%%%%%%%%%%%%%%%%%
%% Elsevier bibliography styles
%%%%%%%%%%%%%%%%%%%%%%%
%% To change the style, put a % in front of the second line of the current style and
%% remove the % from the second line of the style you would like to use.
%%%%%%%%%%%%%%%%%%%%%%%

%% Numbered
%\bibliographystyle{model1-num-names}

%% Numbered without titles
%\bibliographystyle{model1a-num-names}

%% Harvard
\bibliographystyle{model2-names.bst}\biboptions{authoryear}

%% Vancouver numbered
%\usepackage{numcompress}\bibliographystyle{model3-num-names}

%% Vancouver name/year
%\usepackage{numcompress}\bibliographystyle{model4-names}\biboptions{authoryear}

%% APA style
%\bibliographystyle{model5-names}\biboptions{authoryear}

%% AMA style
%\usepackage{numcompress}\bibliographystyle{model6-num-names}

%% `Elsevier LaTeX' style
%\bibliographystyle{elsarticle-num}
%%%%%%%%%%%%%%%%%%%%%%%

%%%%%%%%%%%%%%%%%%%%%%%%%%%%%%%%%%%%%%%%%%%%%%
%%                                          %%
%% Uncomment next line to change            %%
%% the type of equation numbering           %%
%%                                          %%
%%%%%%%%%%%%%%%%%%%%%%%%%%%%%%%%%%%%%%%%%%%%%%
%\numberwithin{equation}{section}
%%%%%%%%%%%%%%%%%%%%%%%%%%%%%%%%%%%%%%%%%%%%%%
%%                                          %%
%% For Assumption, Axiom, Claim, Corollary, %%
%% Lemma, Theorem, Proposition, Hypothesis, %%
%% Fact                                     %%
%% use \theoremstyle{plain}                 %%
%%                                          %%
%%%%%%%%%%%%%%%%%%%%%%%%%%%%%%%%%%%%%%%%%%%%%%
% \theoremstyle{plain}
\newtheorem{axiom}{Axiom}
\newtheorem{postulate}{Postulate}
\newtheorem{claim}[axiom]{Claim}
\newtheorem{theorem}{Theorem}
\newtheorem{lemma}[theorem]{Lemma}
\newdefinition{remark}{Remark}
\newdefinition{definition}{Definition}
\newproof{proof}{Proof}
% \newtheorem*{fact}{Fact}
%%%%%%%%%%%%%%%%%%%%%%%%%%%%%%%%%%%%%%%%%%%%%%
%%                                          %%
%% For Definition, Example, Remark,         %%
%% Notation, Property                       %%
%% use \theoremstyle{remark}                %%
%%                                          %%
%%%%%%%%%%%%%%%%%%%%%%%%%%%%%%%%%%%%%%%%%%%%%%
% \theoremstyle{remark}
% \newtheorem{definition}[theorem]{Definition}
% \newtheorem*{example}{Example}

\newcommand{\proofofref}{}
\newproof{zproofof}{Proof of \proofofref}
\newenvironment{proofof}[1]
 {\renewcommand{\proofofref}{#1}\zproofof}
 {\endzproofof}

%%%%%%%%%%%%%%%%%%%%%%%%%%%%%%%%%%%%%%%%%%%%%%
%% Local Definitions definitions:           %%
%%%%%%%%%%%%%%%%%%%%%%%%%%%%%%%%%%%%%%%%%%%%%%
\DeclareMathOperator{\E}{\mathbb{E}}
\DeclareMathOperator{\Var}{\mathrm{Var}}
\DeclareMathOperator{\Cov}{\mathrm{Cov}}
\DeclareMathOperator{\Tr}{\mathrm{Tr}}
\newcolumntype{L}{>{\centering\arraybackslash}p{0.085\linewidth}}

\begin{document}

\begin{frontmatter}

  \title{Algorithm for Calculating Lightning Node Fees}

  %   %% Group authors per affiliation:
  \author[1]{Cal Abel\texorpdfstring{\footnote{Declarations of interest: none}}}\corref{cor1}
  \address[1]{2600 Century Parkway, Ste. 100, Atlanta, GA, USA}

  %   %% include affiliations in footnotes:
  \author[1]{Skylane Engineering, LLC}
  \ead[url]{https://skylaneengineering.com}

  \cortext[cor1]{Corresponding author}
  \ead{crabel@skylaneengineering.com}

  %   %%%%%%%%%%%%%%%%%%%%%%%%%%%%%%%%%%%%%%%%%%%%%%
  %   %% Abstract                                 %%
  %   %%%%%%%%%%%%%%%%%%%%%%%%%%%%%%%%%%%%%%%%%%%%%%
  \begin{abstract}
    This paper outlines a simple strategy to create channels on a Lightning Network routing node that will tend toward a balanced configuration.
    The algorithm is derived from the principles of statistical mechanics and uses the Helmholtz potential to dynamically assess either inbound or outbound fees.
  \end{abstract}

  %   \begin{keyword}

  %   \end{keyword}

\end{frontmatter}

\linenumbers

%%%%%%%%%%%%%%%%%%%%%%%%%%%%%%%%%%%%%%%%%%%%%%%%%%%%%%%%%%%%%%%%%%%%%%%%%%%%%%%%%%%%%%%%%%%%%%%%%%%
%%% Introduction                                                                                %%%
%%%%%%%%%%%%%%%%%%%%%%%%%%%%%%%%%%%%%%%%%%%%%%%%%%%%%%%%%%%%%%%%%%%%%%%%%%%%%%%%%%%%%%%%%%%%%%%%%%%
\section{Introduction}
The objective of this paper is to lay out the structure of an algorithmic fee structure for routing channel in the Lightning Network.
With current technology, channels are limited in their ability to dynamically assess fees based on channel balance.
What we propose is a method by which a node will use the fee structure to become self balancing, and charging a higher fee for transactions that are routed that take a channel out of balance.

To do this wee need to first formally define a metric for channel balancing.
This is done by treating the sats locked up in a channel as a microcanonical ensemble.
This will provide us with a combinatorial definition of entropy, which when maximized will define a channel being balanced.

The overall approach of this method is to set up a fee structure using the equivalent of the Helmholtz potential.

\section{Defining the Ensemble}
We begin by considering a node in isolation from the rest of the network.
When a node is created traditionally created it is an entirely self funded channel.

\subsection{Ensemble of 1 Satoshi}
Let's look at a single sat channel.
The sat can be in either the inbound state, $\Ket{I}$, or an outbound state, $\Ket{O}$.
Because the sat exists classically a pure state cannot exist as a superposition,
\begin{equation}
  \Ket{\psi} = c_I\Ket{I} + c_O\Ket{O} \; \left\{c_I,c_O\right\} \in \left\{0,1\right\} \textrm{ and } c_I \neq c_O \nonumber
\end{equation}

For this single satoshi we will define its potential relative to the direction that it can be spent.
Thus the sat in $\Ket{I}$ will have a $(+)$ potential and in $\Ket{O}$ a $(-)$ potential.
Because the magnitude value of the satoshi is symmetric we can define the the Hamiltonian as,
\begin{equation}
  \hat{H} =   \left[ {\begin{array}{cc}
          u_0 & 0    \\
          0   & -u_0 \\
        \end{array} } \right].\nonumber
\end{equation}
Thus,
\begin{eqnarray}
  \hat{H}\Ket{I} & = & u_0\Ket{I} \textrm{ and} \nonumber\\
  \hat{H}\Ket{O} & = & -u_0\Ket{O}. \nonumber
\end{eqnarray}

\subsection{Ensemble of Many Satoshi}
We define a channel with $M$ total satoshi and describe the occupancy of the inbound state as $m_I$ and the outbound state as $m_O$.
With the channel so defined we want to to now count the number of different ways that the channel, its multiplicity which is
\begin{equation}
  W = \frac{M!}{m_I!m_O!}. \nonumber
\end{equation}
And define its logarithm as,
\begin{equation}
  S = \log W. \nonumber
\end{equation}
Which for large $M$, and applying the Stirling approximation simplifies to,
\begin{equation}
  S = \sum_j m_j \log\frac{M}{m_j}.\label{eq:1}
\end{equation}
Equation \ref{eq:1} represents the measure of the channels size and, when maximized its balance

\subsection{The Canonical Ensemble}
Because a channel is fixed in its size, the total number of satoshi are invariant with time.
If we bring the channel onto the network where other channels can access it to route payments, we can consider it as an ensemble of a fixed size brought into contact with a reservoir.
This is the canonical ensemble, whose most probable density matrix is,
\begin{equation}
  \hat{\rho} = \frac{1}{Z}e^{-\beta M \left\lvert \hat{H} \right\rvert}. \nonumber
\end{equation}
Where $Z$ is the normalization constant and is also called the partition function.

\subsection{Perturbations form Equilibrium}
The equilibrium state occurs when the channel is balanced thus the probability of finding a satoshi in either the inbound or outbound channels is $p_I=p_O=1/2$.
This isn't very interesting mathematically so we will consider small perturbations around the equilibrium state.
We recall for the canonical ensemble the Helmholtz potential is,
\begin{equation}
  F \equiv - \frac{1}{\beta}\log Z = \Braket{U} - TS. \nonumber
\end{equation}
Where, $T \equiv 1/\beta$.

For a small perturbation,
\begin{equation}
  F + \delta F = \Braket{U} + \left\lvert \delta \Braket{U} \right\rvert - T(S + \delta S).\footnote{
    We take the absolute value of the perturbation because our definition of the sign of the Hamiltonian eigenvectors was entirely arbitrary.
  } \nonumber
\end{equation}
Which reduces to,
\begin{equation}
  \delta F =  \left\lvert \delta \Braket{U} \right\rvert - T \delta S. \nonumber
\end{equation}

Finally if we take the difference between two perturbations,
\begin{equation}
  \Delta F =  \left\lvert \Delta \Braket{U} \right\rvert - T \Delta S. \label{eq:2}
\end{equation}

We take equation \ref{eq:2} as the difference in the potential between the two near equilibrium channel configurations.
Recalling that the Helmholtz potential represents the amount of available work, we will take this as the available potential for extracting a fee.

The $\left\lvert \Delta \Braket{U} \right\rvert$ represents the traditional channel ppm fee.
However there is an additional fee for moving the channel farther from balance and for bringing it closer to balance, $- T \Delta S$.
It is important to note that the derivation did not include a per transaction fee.

Because equation \ref{eq:2} can be negative, we will exclude the case where the channel operator would pay to balance the channel.
The transaction fee is thus,

\boxed{
  \begin{aligned}
     & \textrm{Fee} = \max(0,\Delta F). \nonumber
  \end{aligned}
}

\section{Measuring Temperature}
Recall in defining the ensemble that we did not consider the time dependent, kinetic, portion of the Hamiltonian.
It is not something that is clearly defined for sats on a lightning channel.

Instead we need to look at how the channel is being used in the entire network.
To do this we need to look at the transaction distribution across the channel or for the node.
This distribution should follow some form of the gamma distribution.
If this provides a reasonable estimation, the rate factor, $\beta$, is the inverse temperature \cite[p.~76]{Gibbs:1902}.

The node operator would then need to determine what the fractional charge for temperature would be.
This in effect becomes an average per transaction fee of the change in the entropy.
It's not clear at this time if the measured channel temperature should be used to assess the fee directly, or if it should be multiplied by some constant.

\section{Conclusion}
Hopefully this will all work...

%%%%%%%%%%%%%%%%%%%%%%%%%%%%%%%%%%%%%%%%%%%%%%%%%%%%%%%%%%%%%%%%%%%%%%%%%%%%%%%%%%%%%%%%%%%%%%%%%%%
%%% Acknowledgements                                                                            %%%
%%%%%%%%%%%%%%%%%%%%%%%%%%%%%%%%%%%%%%%%%%%%%%%%%%%%%%%%%%%%%%%%%%%%%%%%%%%%%%%%%%%%%%%%%%%%%%%%%%%
% \section*{Acknowledgements}


\bibliography{paper.bib}

\end{document}
